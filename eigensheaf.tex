\documentclass[./main.tex]{subfiles}
\begin{document}

The previous section was all on the automorphic side.
Finally, we make contact with the Galois side.

\begin{prop}[Geometric Satake]

  Consider the spherical Hecke category \[
    \SPH_G := D\MOD(\HECKE^\LOC)
  \]
  where $\HECKE^\LOC := G\bbrkt{t} \backslash G((t)) / G\bbrkt{t}$.
  This is the version of $\SPH_{G,x}$ for $X = \bA^1, x = 0$.
  \footnote{
    We don't use the projectivity of $X$ for geometric Satake.
    Indeed, since geometric Satake is \emph{local}
    it should be independent of the choice of curve $X$.
  }
  Then the convolution product on $\SPH_G$ 
  makes $\SPH_G$ into a rigid tensor category.
  \footnote{
    Possible question someone might ask : 
    is the convolution product ``commutative'' like in
    the arithmetic case for $\GL_n$?
    The categorification of ``commutative'' is
    symmetric monoidal.
    The answer is yes, but requires hard geometry.
    See \cite[Section 5.2]{Zhu-16}.
  }
  There exists a (faithful exact) tensor functor $F : \SPH_G \to \VEC_k$ 
  which induces an equivalence of tensor categories 
  \[
    \SPH_G \simeq \REP_k G^L , \cS_V \mapsfrom V
  \]
  via the Tannaka reconstruction theorem.
  \footnote{
    If anyone asks :
    roughly speaking,
    this recovers the classical Satake isomorphism
    by taking Grothendieck rings on both sides.
    RHS has basis by dominant coweights $X_\bullet(T)^+$.
    See \cite[Section 5.6]{Zhu-16}.
  }
  \footnote{
    If anyone asks :
    the reason we need to use fpqc quotients instead of
    just prestack quotients is that
    the geometry of the affine Grassmannian 
    $\GR_G := G((t)) / G\bbrkt{t}$ plays an important role in
    proving geometric Satake.
    If we didn't sheafify, 
    the space would simply be wrong.
  }
  \footnote{
    If anyone asks :
    Tannaka reconstruction can be seen as a generalisation of
    Pontrjagin duality to affine algebraic groups.
    For non-commutative groups,
    instead of characters, i.e. 1-dimensional reps,
    one needs all simple reps,
    which is why the tensor category of finite reps serves as
    the ``dual'' to the group.
  }
\end{prop}
The fibers of $\HECKE^\LOC_X \to X$ are 
the local Hecke correspondences $\HECKE^\LOC_x$ at each point $x \in X(k)$,
\begin{cd}
	{\HECKE^\LOC_X} & {\HECKE^\LOC_x} & {\HECKE^\LOC} \\
	X & \PT
	\arrow[from=1-1, to=2-1]
	\arrow["x"{description}, from=2-2, to=2-1]
	\arrow[from=1-2, to=1-1]
	\arrow[from=1-2, to=2-2]
	\arrow["\lrcorner"{anchor=center, pos=0.125, rotate=-90}, draw=none, from=1-2, to=2-1]
	\arrow["\simeq"{description}, draw=none, from=1-2, to=1-3]
	\arrow["{\text{choice of $\bC\bbrkt{t} \simeq \cO^\wedge_x$}}"{description}, shift left=5, draw=none, from=1-2, to=1-3]
\end{cd}
and given a choice of uniformizer at $x$, 
we have $\SPH_G \simeq \SPH_{G,x}$ as tensor categories.
We take on faith that this idea allows one to
take $\cS \in \SPH_G$ and extend it across $\HECKE^\LOC$ in the $X$-direction,
\footnote{
  An account of this is given in \cite[Section 4.2]{Gomez}.
  However, it is somewhat unsatisfactory because the choice of a point 
  $x \in X(k)$ is used whereas the global Hecke action
  doesn't mention any point in particular.
  A fix should be 
  $\HECKE^\LOC_X \simeq X^\wedge \times^{\AUT D^\wedge} \HECKE^\LOC$
  where $X^\wedge$ is the moduli space of formal parameters of $X$
  and $\AUT D^\wedge$ is the pro-algebraic group of
  automorphisms of the formal disk preserving $0$.
}
giving a functor 
\begin{align*}
  \SPH_G &\to \SPH_{G,X} \\
  \cS &\mapsto \tilde{\cS}
\end{align*}
This gives us \emph{two} actions of $\SPH_G \simeq \REP_k G^L$
on $D\MOD(\BUN_G)$.
\begin{enumerate}
  \item On the automorphic side : 
  we let $\SPH_G$ act through all points at once via $\SPH_{G,X}$
  \begin{align*}
    \SPH_{G} \times D\MOD(\BUN_G) \to D\MOD(X \times \BUN_G) \\
    \cS , \cF \mapsto H(\tilde{\cS} , \cF)
  \end{align*}
  \item On the Galois side :
  given a $G^L$ local system $\si$ on $X$,
  we can pullback representations of $G^L$ along $\si$ : 
  \begin{cd}
    {X_\DR} & {BG^L} & \rightsquigarrow & {D\MOD(X)} & {\REP_k G^L}
    \arrow["{\si^*}"', from=1-5, to=1-4]
    \arrow["\si"', from=1-1, to=1-2]
  \end{cd}
  This sends a representation $V$ to the associated bundle $V_\si$
  which has a connection via $\si$.
  So we have the action : 
  \begin{align*}
    \REP_k G^L \times D\MOD(\BUN_G) &\to D\MOD(X \times \BUN_G) \\
    V , \cF &\mapsto V_\si \boxtimes \cF
  \end{align*}
\end{enumerate}
Asking for these two actions to coincide is (roughly) the definition
of a Hecke eigensheaf.
\begin{dfn}
  
  Let $\si \in \LOCSYS_{G^L} X (k)$ be a $G^L$ local system on $X$.
  Then a \emph{Hecke eigensheaf with eigenvalue $\si$} is a
  $\cF \in D\MOD(\BUN_G)$ equipped with isomorphisms
  for each $V \in \REP_k G^L$,\footnote{
    Technically, there are more conditions to
    make the action ``agree''.
    See \cite[Section 4.3]{Gomez}.
  }
  \[
    H(\tilde{\cS_V} , \cF) \simeq V_\si \boxtimes \cF
  \] 
\end{dfn}

\end{document}