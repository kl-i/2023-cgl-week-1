\documentclass[./main.tex]{subfiles}
\begin{document}

We fix a base field $k$ of characteristic zero 
and a smooth projective curve $X$ over $k$.
At some point, this will break can we need to use $k = \bC$.
I now describe some foundational constructions
which is needed to dscribe the Galois and automorphic side of
categorical geometric Langlands. 

\textbf{Idea 1 : Functor of points}

Define the category of affine schemes $\AFF_k$ as 
the opposite of the category of (commutative) $k$-algebras.
For each $k$-algebra $A$, 
we use $\SPEC A$ to denote the corresponding object in $\AFF_k$.
The arrows are reversed because philosophically,
we are viewing each $A$ as the ring of functions on $\SPEC A$.

\begin{dfn}

  The category of \emph{presheaves over $k$} 
  is defined as
  \[
    \PSH_k := \FUN(\AFF_k , \SET^\OP)
  \]
  We denote the Yoneda embedding with $\SPEC$
  so that given a $k$-algebra, $\SPEC A \in \PSH_k$.
  We refer to $A$ as the ring of functions of $\SPEC A$.
\end{dfn}

The idea behind this definition is that
every ``space'' in algebraic geometry should be determined by
how affine schemes map into them.
For example, the affine line $\bA^1$ is defined as the 
forgetful functor $\AFF_k \to \SET^\OP$,
sending each affine scheme $\SPEC A$ to its ring of functions $A$.

The category $\PSH_k$ is nice in the sense that
it shares many categorical properties with $\SET$
so that any construction you can think of in $\SET$ can be done in $\PSH_k$.
As an example, we introduce the formal disk.

Intuitively, the formal disk in $\bA^1$ around $0$
is the space of points in $\bA^1$ infinitesimally close to $0$.
Although the idea of ``infinitesimally close'' is
ill-defined in analysis, the notion of nilpotence works 
for algebraic geometry.
We first define the quotient of $\bA^1$ by 
the relation of two points being infinitesimally close.

\begin{dfn}[de Rham space]

  The \emph{de Rham space of $\bA^1$} is defined as the presheaf
  such that for any affine scheme $\SPEC A$,
  \[
    \bA^1_\DR(A) := \bA^1(A_\RED)
  \]

\end{dfn}

For any affine $S$, we have a closed embedding $S_\RED \to S$.
Restricting along this embedding turns points $S \to \bA^1$
into points $S \to \bA^1_\DR$.
This defines a projection 
\[
  \pi : \bA^1 \to \bA^1_\DR
\]
Here's an example of two points which get identified under this projection.
Consider $t, -t \in k[t]/(t^2) = \bA^1(k[t]/(t^2))$,
which represents two tangent vectors at $0$.
Then $\pi(t) = 0 = \pi(-t) \in \bA^1((k[t]/(t^2))_\RED)$.
This makes sense because the geometric intuition is that
tangent vectors are in the first order neighbourhood of $0$.
We now define the formal disk.

\begin{dfn}

  The \emph{formal disk} is defined 
  by the pullback diagram : 
  \begin{cd}
    {\bA^1} & {D^\wedge} \\
    {\bA^1_\DR} & {\SPEC k}
    \arrow["\pi"', from=1-1, to=2-1]
    \arrow["0", from=2-2, to=2-1]
    \arrow[from=1-2, to=1-1]
    \arrow[from=1-2, to=2-2]
    \arrow["\lrcorner"{anchor=center, pos=0.125, rotate=-90}, draw=none, from=1-2, to=2-1]
  \end{cd}

\end{dfn}

Fiber products in $\PSH_k$ are very concrete.
Simply take fiber products ``pointwise''.
This means that for each affine $A$,
we have the pullback square : 
\begin{cd}
  {\bA^1(A)} & {D^\wedge(A)} \\
  {\bA^1_\DR(A)} & {(\SPEC k)(A)}
  \arrow["\pi"', from=1-1, to=2-1]
  \arrow["0", from=2-2, to=2-1]
  \arrow[from=1-2, to=1-1]
  \arrow[from=1-2, to=2-2]
  \arrow["\lrcorner"{anchor=center, pos=0.125, rotate=-90}, draw=none, from=1-2, to=2-1]
\end{cd}
This is easily computed : 
a map $f : \SPEC A \to \bA^1$ is exactly a function $f \in A$
and $(\SPEC k)(A)$ is singleton so we are looking at exactly
$f \in A$ which become zero in $(\bA^1_\DR)(A) = A_\RED$.
In other words, $D^{\wedge}(A) = N_A$ the nilradial of $A$.

Here is another definition of $D^\wedge$ which shows how nice $\PSH_k$ is.
We have a sequence of close embeddings into $\bA^1$ : 
\begin{cd}
  {\SPEC k[t]/(t)} & \cdots & {\SPEC k[t]/(t^{n+1})} & \cdots & {\bA^1}
	\arrow[from=1-1, to=1-2]
	\arrow[from=1-2, to=1-3]
	\arrow[from=1-3, to=1-4]
	\arrow[from=1-4, to=1-5]
\end{cd}
In $\PSH_k$ we can take the ``union'' i.e. 
colimit $\SPF k\bbrkt{t} := \COLIM_{n} \SPEC k[t]/(t^{n+1})$
of this system of closed embeddings.
What does this look like? 
Again, we do colimits ``pointwise'',
that is to say, 
for each affine $\SPEC A$,
the above corresponds to subset of $A$
\begin{cd}
  {\set{f \in A \st f = 0}} & \cdots & {\set{f \in A \st f^{n+1} = 0}} & \cdots & A
	\arrow["\subseteq", from=1-1, to=1-2]
	\arrow["\subseteq", from=1-2, to=1-3]
	\arrow["\subseteq", from=1-3, to=1-4]
	\arrow["\subseteq", from=1-4, to=1-5]
\end{cd}
We simply take the colimit, which is just the union.
We've recovered $D^\wedge(A)$!
What we've just shown is the following isomorphism in $\PSH_k$ : 
\[
  \SPF k\bbrkt{t} \simeq D^\wedge
\]

\textbf{Idea 2 : Quasi-coherent sheaves on presheaves}

We will need to talk about sheaves on spaces more general than schemes
so let me just say a few words about quasi-coherent sheaves.

There is actually a sensible definition of
quasi-coherent sheaves on \emph{any} presheaf.
(For talk, I won't bore you with the technical definition and
refer to the notes here.)
\begin{dfn}
  
  Let $X \in \PSH_k$.
  A quasi-coherent sheaf $\cF$ on $X$ consists of the following data : 
  \begin{itemize}
    \item for each $x : \SPEC A \to X$, an $A$-module $\cF_x$.
    This is called the fiber of $\cF$ at $x$.
    \item for each $f$ below on the left we have a $f^*$ :
    \begin{cd}
      {\SPEC B} & X & \rightsquigarrow & {\cF_y} & {\cF_x} \\
      {\SPEC A} &&& {\text{}} & {\text{}}
      \arrow["x"', from=2-1, to=1-2]
      \arrow["y", from=1-1, to=1-2]
      \arrow["f"', from=1-1, to=2-1]
      \arrow["{f^*}"', from=1-5, to=1-4]
      \arrow["{\text{in $A\MOD$}}"{description}, draw=none, from=2-4, to=2-5]
    \end{cd}
    Furthermore,
    we have \emph{quasi-coherence},
    meaning $f^* : \cF_x \to \cF_y$ induces $\cF_x \otimes_A B \simeq \cF_y$.
    \item for each $f, g$ below on the left we have 
    \begin{cd}
      {\SPEC C} &&& {\cF_z} \\
      {\SPEC B} & X & {\rightsquigarrow} && {\cF_y} \\
      {\SPEC A} &&& {\cF_x} & {\text{}}
      \arrow["x"', from=3-1, to=2-2]
      \arrow["y", from=2-1, to=2-2]
      \arrow["z", from=1-1, to=2-2]
      \arrow["f"', from=2-1, to=3-1]
      \arrow["g"', from=1-1, to=2-1]
      \arrow["{(fg)^*}"{description}, from=3-4, to=1-4]
      \arrow["{f^*}"', from=3-4, to=2-5]
      \arrow["{g^*}"', from=2-5, to=1-4]
    \end{cd}
  \end{itemize}
  Morphisms of quasi-coherent sheaves are defined fiberwise.
  The collection of quasi-coherent sheaves on $X$ naturally forms
  an abelian category $\QCOH X$.
\end{dfn}
The above definition is sensible in the sense that :
\begin{enumerate}
  \item $\QCOH(\SPEC A) \simeq A\MOD$ for any affine $\SPEC A$.
  \item for a morphism $f : X \to Y$ of presheaves,
  one can construct a functor $f^* : \QCOH Y \to \QCOH X$
  \item For $X$ a scheme and $\cU = \coprod_{i \in I} U_i$
  a Zariski cover by affine opens,
  pulling back along $j : \cU \to X$ gives an equivalence 
  \[
    \QCOH X \map{j^*}{\sim} \QCOH^*(X , \cU)
  \]
  where the latter is
  descent data of modules for the cover $\cU$.
  This recovers the usual definition of quasi-coherent sheaves
  on a scheme.
\end{enumerate}

\textbf{Idea 3 : de Rham space}

We've seen the de Rham space construction
when we defined the formal disk.
I need to explain the de Rham space a bit more
because this is how we're gonna get both local systems and $D$-modules.

One can see that the de Rham space construction
applies to all presheaves.
The general definition is identical.
This in particular applies to our smooth projective curve $X$.
Recall I said for motivation that 
$X_\DR$ looks like $X$ where one identifies all infinitesimally close points.
Making an analogy to sets,
whenever we have an equivalence relation $R \rightrightarrows X$,
we can form the quotient $X \to X / R$ and recover the equivalence relation
as $R \simeq X \times_{X / R} X$.
Again, by the amazing categorical properties of $\PSH_k$,
we can play this game with the projection $\pi : X \to X_\DR$.
We make the following definition. 

\begin{dfn}[Infinitesimal Groupoid of $X$]
  
  The \emph{infinitesimal groupoid of $X$} is defined by
  the fiber product diagram : 
  \begin{cd}
    {\widehat{\De}} & X \\
    X & {X_{\DR}}
    \arrow[from=1-1, to=2-1]
    \arrow[from=2-1, to=2-2]
    \arrow[from=1-1, to=1-2]
    \arrow[from=1-2, to=2-2]
    \arrow["\lrcorner"{anchor=center, pos=0.125}, draw=none, from=1-1, to=2-2]
  \end{cd}
\end{dfn}
One can then show the following : 
\begin{enumerate}
  \item by a similar argument to how we showed $D^\wedge \simeq \SPF k\bbrkt{t}$,
  one can show that the union of the $n$-order neighbourhoods of 
  the diagonal of $X$ recovers the infinitesimal groupoid.\[
    \COLIM_{n} \De^{(n)} \simeq \widehat{\De}
  \]
  \item We have a coequalizer diagram in $\PSH_k$\[
    \widehat{\De} \rightrightarrows X \to X_\DR
  \]
  This affirms our intuition that $X_\DR$ is $X$
  with infinitesimally close points identified.
  \footnote{
    This actually uses formal smoothness of $X$.
  }
  \item Quasi-coherent sheaves on $X_\DR$ are the same as
  quasi-coherent sheaves on $X$ which ``respect the equivalence relation
  $\widehat{\De}$'' in the sense that
  such $\cF \in \QCOH X$ are equipped with
  isomorphisms $\tau_{x,y} : \cF_x \simeq \cF_y$
  for any pair $(x,y)$ of infinitesimally close points
  and that these isomorphisms needs to be
  ``reflexive and transitive''.
  The formal way of saying this is quasi-coherent sheaves on $X$
  equipped with descent data w.r.t. the groupoid $\widehat{\De}$.
  
  In fact, one can show that descent data w.r.t. $\widehat{\De}$ on
  a quasi-coherent sheaf $\cF$ on $X$
  is equivalent to giving a flat connection 
  $\De : \cF \to \cF \otimes_{\cO_X} \Om^1_{X/k}$.
  In particular, this gives the definition 
  \begin{dfn}[D-modules on a presheaf]
    
    Let $X$ be a presheaf.
    Then the category of $D$-modules on $X$ is defined as
    \[
      D\MOD(X) := \QCOH(X_\DR)
    \]
  \end{dfn}
  \item More generally than quasi-coherent sheaves,
  via pullback along $\pi : X \to X_\DR$,
  the category of presheaves over $X_\DR$ is equivalent to
  the category of presheaves over $X$ 
  which ``respect the equivalence relation $\widehat{\De}$''.
  \footnote{
    Again, the formal way of saying this is 
    ``equipped with descent data w.r.t the groupoid $\widehat{\De}$''.
  }
\end{enumerate}

Points (3) and (4) combined say that
objects living over $X_\DR$ should be thought of as
objects on $X$ equipped with a connection.
In particular, a quasi-coherent sheaf $\cF$ on $X_\DR$
whose underlying quasi-coherent sheaf on $X$ is a 
vector bundle of rank $n$ should be thought of as a $n$-dimensional 
representation of the fundamental group of $X$.
This is literally true when $k = \bC$ via the equivalence
between 
\begin{enumerate}
  \item Vector bundles on $X$ equipped with a flat connection.
  This gives an isomorphism of fibers when points are
  ``infinitesimally close''.
  \item Locally constant sheaves on $X$ valued in 
  (finite dimensional) $\bC$-vector spaces.
  This gives isomorphism of fibers when points are related by
  being in a common open that is small enough.
  \item Functors from the fundamental groupoid of $X$ into 
  the category of (finite dimensional) vector spaces over $\bC$.
  This gives isomorphism of fibers when points are related by
  a path-up-to-homotopy.
\end{enumerate}

For the Galois side of geometric Langlands,
we will want group morphisms $\pi_1(X , x) \to G^L$ where
$G^L$ is the Langlands dual of a reductive group $G$ over $k$.
The above analogies then suggest that
the correct formulation of the Galois side is 
\emph{$G^L$ bundles on $X$ equipped with a connection}.
\footnote{
  A precise formulation of this without using de Rham spaces
  can be found in \cite{BB-93}.
}
This brings us to the question : how to define
$G^L$ bundles on something like $X_\DR$ ?
A more important question is : 
can we also make the collection of $G^L$ bundles on $X_\DR$
into an object of algebraic geometry i.e. a presheaf over $k$? 

\textbf{Idea 4 : Stack quotients by group actions}

Let me first say that the standard definition of
$G$ bundles on a scheme $X$
generalise to the setting where 
$G$ is any group object in $\PSH_k$ and $X$ is any presheaf.
\footnote{
  Another name for $G$ bundles is \emph{$G$ torsors}.
}
I leave the definition to the notes.
\begin{dfn}

  Let $G$ be a group object in $\PSH_k$
  and $X$ a general presheaf.
  Then a $G$ bundle on $X$ is 
  a morphism $P \to X$ where
  \begin{enumerate}
    \item $P$ is equipped with a right $G$-action, $X$ the trivial action, 
    and $P \to X$ is $G$-equivariant
    \item for all points $x : S \to X$ where $S$ is affine,
    there exists an fpqc affine $T \to S$ such that
    $T \times_X P$ is isomorphic to $T \times G$ as $G$-actions.
  \end{enumerate}
\end{dfn}

And now we try to define the \emph{classifying space of $G$ bundles}.
This is related to taking quotients by group actions
so I will describe the general procedure.
If I had more time, I would do a more detailed motivation for stacks.
but we're short on time here so let me just give you the answer
people came up with and some intuition to make it believable.
% \begin{dfn}[Attempt 1 at $\bB G$]
  
%   Define $\bB G$ by the functor \[
%     \bB G (A) := \text{ set of $G$ bundles on $\SPEC A$}
%   \]
% \end{dfn}
% What's the issue with this?
% Well, it's not functorial!
% One can see this at the level of sets already.
% Take the following diagram in $\SET$.
% \begin{cd}
% 	{\set{b}} & {\set{(b,c)}} & {\set{((b,c),d)}} & {\set{(b,d)}} \\
% 	{\set{a}} & {\set{c}} & {\set{d}}
% 	\arrow[from=1-2, to=1-1]
% 	\arrow[from=1-3, to=1-2]
% 	\arrow["g"', from=2-3, to=2-2]
% 	\arrow["f"', from=2-2, to=2-1]
% 	\arrow[from=1-1, to=2-1]
% 	\arrow[from=1-2, to=2-2]
% 	\arrow[from=1-3, to=2-3]
% 	\arrow["\lrcorner"{anchor=center, pos=0.125, rotate=-90}, draw=none, from=1-2, to=2-1]
% 	\arrow["\lrcorner"{anchor=center, pos=0.125, rotate=-90}, draw=none, from=1-3, to=2-2]
% 	\arrow["\neq"{description}, draw=none, from=1-4, to=1-3]
% \end{cd}
% The point is that $(fg)^*$ and $g^* f^*$ are not equal
% but only isomorphic.\footnote{
%   If you really want this to be a counter example,
%   you can define these singletons as constant presheaves
%   and use the trivial group as $G$.
%   Then this demonstrates pullback of $G$ bundles
%   fails to be functorial.
% }
% ``Okay, then let's take isomorphism classes'' you might say.
% \begin{dfn}[Attempt 2 at $\cB G$]
  
%   Define $\bB G$ by the functor \[
%     \bB G (A) := \text{ set of isomorphism classes of $G$ bundles on $\SPEC A$}
%   \]
% \end{dfn}
% Now this is strictly functorial.
% So what's the issue?
% The issue is \emph{descent}.

\begin{dfn}[Quotient stack]
  
  Let $X$ be a presheaf and $G$ a group presheaf.
  Then the \emph{(fpqc) quotient stack X / G} is defined
  as the ``functor'' that sends each affine $S$ to 
  the \emph{groupoid} of diagrams
  \begin{cd}
    P & X \\
    {S}
    \arrow["{\text{G bundle}}"', from=1-1, to=2-1]
    \arrow["{G\text{ equiv}}", from=1-1, to=1-2]
  \end{cd}
\end{dfn}
Intuition for this definition : 
\begin{enumerate}
  \item Analogy with group actions on sets : 
  Assume momentarily that $X$ and $G$ and $S$ are sets.
  Then a map $S \to X$ is equivalent to a diagram
  \begin{cd}
    S \times G & X \\
    {S}
    \arrow["{\text{G bundle}}"', from=1-1, to=2-1]
    \arrow["{G\text{ equiv}}", from=1-1, to=1-2]
  \end{cd}
  Intuitively, 
  ``the points in $S$ maps to points in $X$ which gives $G$ orbits
  in the obvious way''.
  The set of such diagrams forms a groupoid.
  Intuitively, a path moves the $S$-family of points in $X$ 
  along their individual orbits.
  \begin{cd}
    {S \times G} & X \\
    S & {S \times G}
    \arrow[from=1-1, to=2-1]
    \arrow[from=2-2, to=2-1]
    \arrow["g"', from=1-1, to=2-2]
    \arrow["x", from=1-1, to=1-2]
    \arrow["{y = g.x}"', from=2-2, to=1-2]
  \end{cd}
  We see that the path connected components of this groupoid
  bijects with the set of $S$-families of $G$ orbits in $X$,
  i.e. maps $S \to X / G$.

  Now back to algebraic geometry, the idea is the same
  except we allow the $S$-family of copies of $G$ to 
  ``vary non-trivially across $S$''.
  This is exactly what a $G$ bundle on $S$ is.
  \item There is a map $X \to X / G$.
  For each point $x : S \to X$ where $S$ is affine,
  note that this is equivalent to a diagram 
  of the above form.
  This says ``points of $X$ give rise to $G$-orbits in $X$''.
  So we send $x \in X(S)$ to the point in $(X / G)(S)$
  which is the trivial $G$ bundle on $S$.
  \item Maps into $X / G$ glue.
  This is a natural generalisation of the fact that
  maps into a fixed scheme glue which
  reconciles with the fact that points of $X / G$ now have
  \emph{possibly non-trivial ``equalities''}.
  \footnote{
    If anyone asks,
    this also explains why we want groupoids.
    If we took isomorphism classes instead, then maps don't glue.
    Give the $\bB \bG_m$ and $\bP^1$ example.
  }
  The fancy way of saying this is that $X / G$ satisfies (fpqc) descent.
  ``Groupoid valued presheaves which satisfy descent''
  are what people mean by \emph{stacks}.
  % For example, given a scheme $S$ and a
  % Zariski cover of affine opens $\coprod_{i \in I} U_i = \cU \to S$, 
  % a map $S \to [X / G]$ is equivalent to
  % a map $\cU \to [X / G]$ that 
  % agrees on double intersections $\cU \times_S \cU$ 
  % \emph{and triple intersections} $\cU \times_S \cU \times_S \cU$.
  \item Taking $X = \SPEC k$ equipped with the trivial action
  gives the \emph{classifying stack of $G$ bundles},
  usually denoted with $BG$.
  \footnote{
    Fingers crossed no one asks about the universal property of
    stack quotient.
    I can't find reference in \cite{lurie-htt}.
  }
  \item $\QCOH X / G$ is equivalent to the category 
  $\QCOH^* (X , X \times G)$ of quasi-coherent sheaves on $X$ equipped with  
  equivariance w.r.t. the action of $G$.
  \footnote{
    Formally, equivariance w.r.t. the action groupoid 
    $G \times X \rightrightarrows X$.
  }
  In particular, $\QCOH(BG) \simeq \REP_k G$.
\end{enumerate}

Now that I've given some reasons for why we want to use stacks,
one can hopefully see that technically speaking 
we now need to redo the foundations,
replacing presheaves with \emph{prestacks}.
This is what \cite{lurie-htt} does for us.
Modulo 500+ pages of technicalities,
the definition is the same except sets are replaced with
sets with paths and paths between paths and so on,
A.K.A. \emph{$\infty$-groupoids}.
\footnote{
  If anyone asks :
  technically, we only need $1$-groupoids and this is what \cite{BD} does.
  But all the latest papers, e.g. by Gaitsgory,
  are done in the language of higher categories as developed by Lurie.
}

\begin{dfn}
  
  The $\infty$-category of \emph{prestacks over $k$} is defined as
  the functor category
  \[
    \PSTK_k := \mathrm{Fun}(\AFF_k , \infty\GRPD^\OP)
  \]
  where $\infty\GRPD$ is the $\infty$-category of $\infty$-groupoids.

  There is a full subcategory of prestacks satisfying 
  (fpqc) descent.
  This is called the category of \emph{stacks over $k$}.
  \[
    \STK_k \subs \PSTK_k
  \]
\end{dfn}

\end{document}