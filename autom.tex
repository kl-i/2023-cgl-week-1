\documentclass[./main.tex]{subfiles}
\begin{document}

Now I describe the automorphic side.
\begin{dfn}[Automorphic side]
  
  The \emph{stack of $G$ bundles on $X$} is defined as 
  \[
    \BUN_G X := \HOM(X , BG)
  \]
  The automorphic side the categorical geometric Langlands 
  is defined as the category 
  \[
    D\MOD(\BUN_G X)
  \]
  I might only write $\BUN_G$ sometimes.
\end{dfn}
The above is the categorical geometric analogue of the
space of (unramified) cusp forms.
Here are some questions you might have :
\begin{enumerate}
  \item Question : Why $\BUN_G X$? 
  Short answer : geometric interpretation of double coset space of adeles.
  Long answer : 
  By ``geometric interpretation'' I mean 
  that the path connected components of the groupoid $\BUN_G(k)$
  gives back the double coset space of adeles.
  First, let me show you why double cosets appear by 
  looking at the example of $G = \GL_1$ and $X = \bP^1$.
  Then $\pi_0 \BUN_G X(k)$ is the set of isomorphism classes of
  $\GL_1$ bundles on $\bP^1$, 
  equivalently line bundles on $\bP^1$.
  There's an easy classification :
  take the standard cover $U_0 + U_1 \to \bP^1$ with $U_i = \bA^1$.
  Then fpqc descent says that a 
  a line bundle $L$ on $\bP^1$
  is equivalently a pair of line bundles $L_i$ on $U_i$ 
  equipped with an isomorphism $\al : L_0 \to E_1$ over $U_0 \cap U_1$.
  But vector bundles on $\bA^1$ can all be trivialised,
  giving us an isomorphism of descent data : 
  \begin{cd}
	L & \leftrightsquigarrow 
  & {L_0} & {\res{L_0}{U_{01}}} & {\res{L_1}{U_{01}}} & {L_1} \\
	&& {\cO_{U_0}} & {\cO_{U_{01}}} & {\cO_{U_{01}}} & {\cO_{U_1}}
	\arrow[from=1-4, to=1-3]
	\arrow[from=1-5, to=1-6]
	\arrow["\alpha", from=1-4, to=1-5]
	\arrow["{s_0}"', from=2-3, to=1-3]
	\arrow["{s_1}"', from=2-6, to=1-6]
	\arrow[from=2-4, to=2-3]
	\arrow[from=2-5, to=2-6]
	\arrow["{s_0}"', from=2-4, to=1-4]
	\arrow["{s_1}"', from=2-5, to=1-5]
	\arrow["{s_1^{-1}\alpha s_0}"', from=2-4, to=2-5]
	\arrow[from=2-3, to=1-3]
  \end{cd}
  So each line bundle $L$ on $\bP^1$ with trivialisations 
  $s_i$ on $U_i$
  is determined up to isomorphism by
  the element $s_1^{-1} \alpha s_0 \in \GL_n(U_{01})$.
  In other words we have the isomorphism below : 
\begin{cd}
	{\pi_0\set{\text{line bundles on $\bP^1$ with $s_0, s_1$}}} & {\GL_1(U_{01})} \\
	{\pi_0(\BUN_{\GL_1}\bP^1)(k)} & {GL_1(U_0)\backslash GL_1(U_{01}) / \GL_1(U_1)}
	\arrow["\sim"', from=1-2, to=1-1]
	\arrow[from=1-1, to=2-1]
	\arrow[from=1-2, to=2-2]
	\arrow["\sim", from=2-2, to=2-1]
\end{cd}
  The set of trivialisations $s_i$ form torsors for $\GL_1(U_i)$ respectively.
  Thus, forgetting the trivialisations gives us the bottom bijection.
  
  Now we show how adeles appear.
  \begin{enumerate}
    \item For a fixed open $U$ of the curve.
    The idea is to replace the cover $U_0, U_1$ by
    $U, \coprod_{x \in X} D_x$ where $D_x$ is the disk around $x \in X(k)$.
    \footnote{
      If anyone asks,
      this works by fpqc descent and the fact that $X$ is finite type.
    }
    One has
    \begin{align*}
      &\set{\text{isomorphism class of line bundles on $X$
      which trivialise on $U$}} \\
      \simeq \,
      & \GL_1(U) \backslash 
      \GL_1(U \cap \coprod_{x \in X} D_x) 
      / \GL_1(\coprod_{x \in X} D_x) \\
      \simeq\,
      & \GL_1(U) \backslash 
      \GL_1 (\prod_{x \in U} \cO_x^\wedge \times \prod_{x \notin U} K^\wedge_x)
      /\GL_1( \prod_{x \in X} \cO_x^\wedge)
    \end{align*}
    \item Now take the union across all opens $U$.
    $\GL_1(U)$ becomes $\GL_1(K)$ \footnote{
      $\GL_1$ separated so maps $\GL_1(U) \subs \GL_1(V)$ for $V \subs U$
    },
    middle term becomes $\GL_1(\bA)$
    and the right hand term $\GL_1(\cO)$
    We obtain what's called the \emph{Weil uniformisation}.
    \[
      \pi_0 \BUN_{\GL_1} X (k) \simeq
      \GL_1(K) \backslash \GL_1(\bA) / \GL_1(\bO_X)
    \]
  \end{enumerate}
  So we see that $\BUN_G$ geometricises the adelic double coset.
  \item Question : why $D$-modules on $\BUN_G X$? 
  Short answer : 
  when $k = \bF_q$, 
  perverse sheaves categorify functions
  % $\overline{\bQ_\ell}$-valued functions
  and when $k = \bC$, 
  $D$-modules (with certain conditions) 
  equivalent to perverse sheaves via Riemann--Hilbert correspondence.

  Long answer :
  I've already touched on the Riemann--Hilbert correspondence.
  It is a generalisation of the equivalence between
  locally constant sheaves and
  vector bundles equipped with flat connections.
  So let me describe how perverse sheaves categorify functions.

  The setting is $k = \bF_q$ and for simplicity again
  let $G = \GL_1$.
  We fix an algebraic closure $\bF_q \subs \overline{\bF}$.
  Let $\cF \in \PERV(\BUN_G , \overline{\bQ_\ell})$.
  I won't tell you what this is, but all you need to know is
  that given a $x : \SPEC \bF_q \to \BUN_G$,
  we can get a $\overline{\bQ_\ell}$-vector space $\cF_{\overline{x}}$
  by pulling back.
  \begin{cd}
    {\cF_{\overline{x}}} \arrow[loop left, "{\FROB^g_x}"] & {\cF_{x}} & \cF \\
    {\overline{x}} \arrow[loop left, "{\FROB^g_x}"] & x & {\BUN_1}
    \arrow[from=2-2, to=2-3]
    \arrow[from=2-1, to=2-2]
    \arrow[squiggly, from=1-3, to=1-2]
    \arrow[squiggly, from=1-2, to=1-1]
  \end{cd}
  Intuitively, the Galois group $\GAL(\bar{\bF_q}/\bF_q)$ acts
  on $\overline{x} \to x$,
  and thus also on $\cF_{\bar{x}}$.
  We can then take the trace of the (geometric) Frobenius.
  All together, we obtain a ring map
  \begin{align*}
    K(\PERV(\BUN_1 , \overline{\bQ_\ell})) 
    &\to \overline{\bQ_\ell}\sqbrkt{\GL_1(K) \backslash \GL_1 \bA / \GL_1 \cO} \\
    \cF & \mapsto (x \mapsto \mathrm{Tr}({\FROB}^g_x , \cF_{\overline{x}}))
  \end{align*}
  This is roughly what people mean by 
  ``perverse sheaves categorify functions''.
  \footnote{
    A precise statement can be found in \cite[Chapter III, Theorem 12.1]{KW}.
  }
  In \cite{D-83}, Drinfeld's proof of geometric Langlands for
  $\GL_2$ constructs Hecke eigenfunctions 
  by first constructing the perverse sheaf,
  then takes trace of Frobenius.

\end{enumerate}

\end{document}