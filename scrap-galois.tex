\documentclass[./main.tex]{subfiles}
\begin{document}

We begin by describing the Galois side.
For motivation, anything I say about the fundamental group
is intentionally vague.

Classically, Galois representations are
representations of the absolute Galois group $\GAL(\bar{K} / K)$.
Let $\bar{\eta} = \SPEC\overline{K} \to \SPEC K = \eta$ 
be the geometric point of $X$ corresponding
to the choice of algebraic closure $\bar{K}$ of $K$.
Let $E$ be the field our representations are over.
Classically, $E$ is $\overline{\bQ}_\ell$ when $k = \bF_q$ 
and $\bC$ when $k = \bC$.
Then an $n$-dimensional Galois represention is a group homomorphism 
\[
  \pi_1^\ET(\eta, \overline{\eta}) := \GAL(\bar{K} / K) \to \GL_n(E)
\]

The first step is to replace $\GAL(\bar{K} / K)$ with the
étale fundamental group of $X$.
It can be defined as follows :
\footnote{
  \href{https://stacks.math.columbia.edu/tag/0BQM}{Stacks Project, Proposition
  58.11.3}
}
\begin{cd}
  {\mathrm{Gal}(\overline{K}/K)} & {\pi_1^\ET(X,\overline{\eta})} \\
  {\mathrm{Gal}(M / K)}
  \arrow[two heads, from=1-1, to=2-1]
  \arrow["\sim"', from=2-1, to=1-2]
  \arrow[from=1-1, to=1-2]
\end{cd}
where $M$ is the compositum of all finite subextensions 
$K \to E \subs \overline{K}$ such that 
\begin{cd}
  {\mathrm{Spec}\,E} & {X^E} \\
  & X
  \arrow[from=1-1, to=2-2]
  \arrow["{\text{normalisation}}"{yshift=3mm}, from=1-1, to=1-2]
  \arrow["{\text{unram. eqv. étale}}", from=1-2, to=2-2]
\end{cd}

We now pass over to topology.
Restrict to $k = E = \bC$.
Then the Riemann existence theorem says
there is an equivalence between the category of finite étale covers of $X$
and that of $X(\bC)$.
From this, one obtains \[
  \pi_1^\ET(X , \overline{\eta}) \simeq \pi_1(X, x)^\wedge
\]
where the latter is the profinite completion of the topological
fundamental group of $X(\bC)$.
We work directly with $\pi_1(X, x)$.

Now, topologically there are three ways of talking about
representations of $\pi_1(X,x)$.
The point is that all three notions are describing the same idea of
giving isomorphisms between fibers along points that are ``close together''
\begin{cd}
	{\mathrm{Fun}(\Pi_1X , \mathrm{Vec}_{\mathbb{C}})} 
  & {\mathrm{Sh}_{\text{l.c.}}(X,\mathbb{C})} 
  & {\mathrm{QCoh}(X , X^{(\infty)})}
	\arrow["\simeq"{description}, draw=none, from=1-2, to=1-3]
	\arrow["\simeq"{description}, draw=none, from=1-1, to=1-2]
\end{cd}
\begin{enumerate}
  \item Functors from the fundamental groupoid of $X$ into 
  the category of (finite dimenstional) vector spaces over $\bC$.
  This gives isomorphism of fibers when points are related by
  a path-up-to-homotopy.
  \item Locally constant sheaves on $X$ valued in 
  (finite dimensional) $\bC$-vector spaces.
  This gives isomorphism of fibers when points are related by
  being in a common open that is small enough.
  \item Vector bundles on $X$ equipped with a flat connection.
  This gives an isomorphism of fibers when points are
  ``infinitesimally close''.
\end{enumerate}
The final description is the one we will focus on.
From now on, we do not need $k$ to be $\bC$ anymore.

Although the notion of points being ``infinitesimally close'' is
ill-defined in analysis, it can be made precise in algebraic geometry
using nilpotence.
To describe this, it is convenient to use the functor of points approach
to algebraic geometry.

\begin{dfn}

  Let $\AFF_k$ denote the opposite of the category of
  commutative $k$-algebras.
  Define the category of \emph{presheaves over $k$}
  to be \[
    \PSH_k := \PSH \AFF_k
  \]
  Let $\AFF_k^\FT$ be the full subcategory of $\AFF_k$
  consisting of finite type affine schemes.
  This induces a fully faithful functor
  \[
    \PSH \AFF_k^\FT \to \PSH \AFF_k
  \]
  We view $\PSH \AFF_k^\FT$ as a full subcategory of $\PSH \AFF_k$.
  Presheaves in here are called \emph{locally finite type}.
\end{dfn}
Practically speaking,
the locally finite type assumption allows us to
do all manipulations with only finite type $k$-algebras.

The curve $X$ lives $\PSH \AFF_k$ by
$X(A) := \HOM(\SPEC A , X)$
and is in fact locally of finite type.
\footnote{TODO}

We now define ``the quotient of $X$ by infinitesimally close points''.
\begin{dfn}
  The \emph{de Rham space} of $X$ is defined as the functor 
  \[
    X_\DR(A) := X(A_\RED)
  \]
  where $A_\RED$ is the quotient of $A$ by its nilradical.

  There is a morphism $\pi : X \to X_\DR$ given by
  restricting any $\SPEC A \to X$ along $\SPEC A_\RED \to \SPEC A$.
\end{dfn}

Since $X$ is locally finite type,
we have that $X$ is formally smooth iff $\pi : X \to X_\DR$ is surjective.
With sets,
a map $X \to Y$ is surjective iff we have a coequalizer diagram : 
\[
  X \times_Y X \rightrightarrows X \to Y
\]
i.e. $Y$ is the quotient of $X$ by the equivalence relation $X \times_Y X$.
The great thing about $\PSH \AFF_k$ is that
it shares many categorical properties
to the category $\SET$ of sets in the sense that 
any construction you can think of in $\SET$ can be done in $\PSH \AFF$.
\footnote{The categorical jargon for this is that $\PSH_k$ is a 
\emph{Grothendieck topos}.}
In particular, 
limits and colimits exist in $\PSH \AFF_k$ and
are computed ``point-wise''.
As an example, for $f : X \to Y$ in $\PSH \AFF_k$,
we have $f$ is an epimorphism in $\PSH \AFF_k$ iff
for all $A \in \AFF_k$, $f : X(A) \to Y(A)$ is an epimorphism in $\SET$.
Applying this, we obtain a coequalizer diagram : 
\[
  X^{(\infty)} := X \times_{X_\DR} X \rightrightarrows X \to X_\DR
\]
$X^{(\infty)}$ here is called the \emph{infinitesimal groupoid} of $X$.
It has the following concrete description : 
since $X$ is separated, the diagonal $X \to X \times X$ is a closed embedding.
Let $I_\De$ be the ideal sheaf of the diagonal
and $X(n) := V(I_\De^{n+1})$ denote the $n$-order neighbourhood of
the diagonal.
These form an inductive system of closed subschemes of $X \times X$.
We then have \[
  \COLIM_{n \in \bN} X^{(n)} \simeq X^{(\infty)}
\]

Given $\cF \in \QCOH X$,
a flat connection on $\cF$ should for each $(x,y) \in (X \times X)(A)$
with $\pi(x) = \pi(y)$, give an isomorphism $\phi_{x,y} : \cF_x \to \cF_y$.
Moreover, we want the identification of fibers to
respect the fact that the equivalence relation $X^{(\infty)}$
is reflexive and transitive.
Making this precise amounts to defining
an \emph{equivariance structure on $\cF$ w.r.t. $X^{(\infty)}$}.

\begin{dfn}
  
  Consider the following diagram : 
  \begin{cd}
    {X \times_{X_\DR} X \times_{X_\DR} X} & {X \times_{X_\DR} X} & {X}
    \arrow["{p_0}"{description}, shift left=2, from=1-2, to=1-3]
    \arrow["{p_{12}}"{description}, from=1-1, to=1-2]
    \arrow["{p_{02}}"{description}, shift right=4, from=1-1, to=1-2]
    \arrow["{p_{01}}"{description}, shift left=4, from=1-1, to=1-2]
    \arrow["{p_1}"{description}, shift right=2, from=1-2, to=1-3]
  \end{cd}
  where we have the projections \begin{itemize}
    \item $p_i : X \times_{X_\DR} X \to X, (x_0 , x_1) \mapsto x_i$
    \item $p_{ij} : X \times_{X_\DR} X \times_{X_\DR} X \to X \times_{X_\DR} X, 
    (x_0 , x_1, x_2) \mapsto (x_i , x_j)$
  \end{itemize}

  Then define the \emph{category $\QCOH^*(X , X^{(\infty)})$ 
  of infinitesimally equivariant quasi-coherent sheaves on $X$} as follows : 
  \begin{itemize}
    \item An object is 
    a quasi-coherent sheaf $\cF \in \QCOH X$
    equipped with a \emph{transition map} 
    $\phi : p_0^* \cF \cong p_1^* \cF$ 
    in $\QCOH( X \times_{X_\DR} X)$ such that
    \begin{enumerate}
      \item $\phi = \id$ when restricted to the diagonal 
      $\De : X \to X \times_{X_\DR} X$
      \item the following diagram commute : 
      \begin{cd}
        & {p_{01}^*p_1^* \mathcal{F}} \\
        {p_{01}^*p_0^*\mathcal{F}} && {p_{12}^*p_0^*\mathcal{F}} \\
        {p_{02}^*p_0^*\mathcal{F}} && {p_{12}^*p_1^*\mathcal{F}} \\
        & {p_{02}^*p_1^* \mathcal{F}}
        \arrow["{p_{01}^*(\phi)}", from=2-1, to=1-2]
        \arrow["{p_{12}^*(\phi)}", from=2-3, to=3-3]
        \arrow["{p_{02}^*(\phi)}"', from=3-1, to=4-2]
        \arrow["\sim"', from=2-1, to=3-1]
        \arrow["\sim", from=1-2, to=2-3]
        \arrow["\sim", from=3-3, to=4-2]
      \end{cd}
    \end{enumerate}
    \item a morphism $\eta : (\cF , \phi) \to (\cG , \psi)$ is
    a morphism $\eta : \cF \to \cG$ in $\QCOH X$ such that
    the following commutes : 
    \begin{cd}
      {p_0^* \cF} & {p_1^* \cF} \\
      {p_0^* \cG} & {p_1^* \cG}
      \arrow["{p_0^*(\eta)}"', from=1-1, to=2-1]
      \arrow["{p_1^*(\eta)}", from=1-2, to=2-2]
      \arrow["\psi"', from=2-1, to=2-2]
      \arrow["\phi", from=1-1, to=1-2]
    \end{cd}
  \end{itemize}
  
\end{dfn}
One can show that for $\cF \in \QCOH X$,
an infinitesimal equivariance structure is the same as
a connection in the classical sense.

In general, for any $Y \in \PSH \AFF_k$ one can define a category 
$\QCOH Y$ of quasi-coherent sheaves on $Y$.
There will be for any $f : X \to Y$ a pullback functor
$f^* : \QCOH Y \to \QCOH X$.
One can then show
\begin{prop}
  
  For $\cF \in \QCOH X_\DR$ the pullback $\pi^* \cF$ has
  an obvious equivariance structure w.r.t. $X^{(\infty)}$.
  This induces an equivalence \[
    \pi^* : \QCOH X_\DR \map{\sim}{} \QCOH^*(X , X^{(\infty)})
  \]
\end{prop}

In general, we have the following.

\begin{prop}

  Define $(\PSH_k / X)^{X^{(\infty)}}$ as
  presheaves over $X$ equipped with equivariance w.r.t $X^{(\infty)}$.
  Then pulling back presheaves over $X_\DR$ along $\pi$
  gives an equivalence of categories \[
    \pi^* : \PSH_k / X_\DR \map{\sim}{} (\PSH_k / X)^{X^{(\infty)}}
  \]
\end{prop}
One should thus think of objects living over $X_\DR$ as
objects on $X$ equipped with a connection.

Finally, we make the step from $\GL_n$ to general reductive $G$.
A rank $n$ vector bundle on $X$ is equivalently a $\GL_n$-torsor on $X$.
It makes sense to talk about $G$-torsors on $X$ for any algebraic group $G$.
In fact, it makes sense to talk about $G$-torsors on any presheaf.
See \cite[Section 0.3.3]{Zhu-16}.

\begin{dfn}
  
  A $G^L$-local system on $X$ is a $G^L$-torsor over $X_\DR$.
\end{dfn}

Let $\LOCSYS_{G^L}(k)$ denote the set of isomorphism classes of 
$G^L$ local systems on $X$.
We would like to give this set some geometric structure.
To do this, a natural first question to ask is
whether one can make a moduli space for $G^L$ torsors.
This is where stacks appear.

\begin{dfn}
  
  Let $G$ be a general algebraic group.
  The \emph{classifying stack of $G$} is defined as the
  follow \emph{groupoid} valued functor \[
    BG(A) := \text{ groupoid of $G$-torsors on $\SPEC A$}
  \] 
\end{dfn}
Technically speaking, $BG$ doesn't live in $\PSH_k$ anymore.
This is one aspect with higher categories appear.
Sets precisely groupoids where there are no non-trivial paths between points.
In this sense, groupoids generalise sets.
Although at this point we do not need to go to higher groupoids,
the framework has been conveniently laid out in \cite{lurie-htt}.
This leads to the following refinement of $\PSH_k$.

\begin{dfn}
  
  The $\infty$-category of \emph{prestacks over $k$} is defined as
  the functor category
  \[
    \PSTK_k := \mathrm{Fun}(\AFF_k , \infty\GRPD^\OP)
  \]
  where $\infty\GRPD$ is the $\infty$-category of $\infty$-groupoids.

  We analogously define 
  $\PSTK_k^\FT := \mathrm{Fun}(\AFF_k^\FT , \infty\GRPD^\OP)$.
\end{dfn}
In particular, we have $BG^L \in \PSTK_k$.
What we want now is a ``space'' which classifies
maps from $X_\DR$ to $BG^L$.
In other words, we would like to form a ``mapping stack''
$\underline{\MAP}(X_\DR , BG^L)$.
Again, with inspiration from $\SET$,
one comes up with the following.
\begin{dfn}
  The mapping stack $\underline{\MAP}(X_\DR , BG^L)$ is defined as \[
    \underline{\MAP}(X_\DR , BG^L)(A) := 
    \MAP(X_\DR \times \SPEC A , BG^L)
  \]

  The \emph{stack of $G^L$ local systems on $X$} is defined to be
  exactly the above \[
    \LOCSYS_{G^L} := \underline{\MAP}(X_\DR , BG^L)
  \]
\end{dfn}
Note that taking $k$-valued points of $\LOCSYS_{G^L}$ gives
the groupoid $\LOCSYS_{G^L}(k)$ of $G^L$-torsors on $X_\DR$.
By connected components of this groupoid, we recover
isomorphism classes of $G^L$ local systems on $X$.


\end{document}