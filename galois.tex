\documentclass[./main.tex]{subfiles}
\begin{document}

I begin by giving an impression of the Galois side.
Let $G$ be a reductive group over $k$
and let $G^L$ denote the Langlands dual.
\footnote{
  Because we are working with $k = \bC$,
  $G^L$ is also over $k$ with root datum
  transpose to that of $G$.
}

In geometric Langlands,
the $L$-parameters will be group morphisms $\pi_1(X,x) \to G^L$.
The definition of the Galois side is as follows.
\begin{dfn}[Galois side]
  
  The \emph{stack of $G^L$ local systems on $X$} is 
  defined as \[
    \LOCSYS_{G^L} := \HOM(X_\DR , BG^L)
  \]
\end{dfn}
Let me give some comments on what this has to do with morphisms 
$\pi_1(X,x) \to G^L$.
\begin{enumerate}
  \item (over de Rham space = connection) 
  $X_\DR$ here is the \emph{de Rham space} of $X$.
  It is the quotient of $X$ by the ``equivalence relation''
  \[
    \widehat{\De} \rightrightarrows X \times X
  \]
  which says $(x,y) \in \widehat{\De}$ iff 
  $x, y$ are ``\emph{infinitesimally close}''.
  $\widehat{\De}$ is called the \emph{infinitesimal groupoid} and
  one can show that \[
    \COLIM_{n} \De^{(n)} \simeq \widehat{\De}
  \]
  where $\De^{(n)}$ is the $n$-order neighbourhood of the diagonal of $X$.
  One can also show that pulling back along 
  the quotient map $\pi : X \to X_\DR$ induces an equivalence
  \begin{cd}
    {\QCOH X_\DR} & {\QCOH^*(X , \widehat{\De})}
    \arrow["{\pi^*}", from=1-1, to=1-2]
    \arrow["\sim"', from=1-1, to=1-2]
  \end{cd}
  where the latter consists of $\cF \in \QCOH X$
  which ``respect the equivalence relation
  $\widehat{\De}$'' in the sense that
  such $\cF \in \QCOH X$ are equipped with
  isomorphisms $\tau_{x,y} : \cF_x \simeq \cF_y$
  for any pair $(x,y)$ of infinitesimally close points
  and that these isomorphisms needs to be
  ``reflexive and transitive''.
  The formal way of saying this is quasi-coherent sheaves on $X$
  equipped with descent data w.r.t. the groupoid $\widehat{\De}$.
  It turns out that descent data w.r.t $\widehat{\De}$ is equivalent to
  giving a flat connection on $\cF$.
  \cite[Theorem 0.4]{BDSeminar}
  This gives us a definition / theorem.
  \begin{dfn}
    The \emph{category of $D$-modules on $X$} is defined to be 
    \[
      D\MOD\,X := \QCOH X_\DR
    \]
  \end{dfn}
  Why is this relevant? 
  It's relevant because over $\bC$ 
  there are three equivalent ways of thinking about
  local systems : 
  \begin{cd}
    {\begin{matrix} \text{(finite) representations of} \\
      \text{fundamental group(oid) of $X$}\end{matrix}} 
    && {\begin{matrix}\text{locally const} \\
       \text{(finite rank) sheaves on $X$}\end{matrix}} \\
    & {\begin{matrix} \text{vector bundles on $X$} \\ 
      \text{equipped with flat connection} \end{matrix}}
    \arrow["\sim"{description}, from=1-1, to=1-3]
    \arrow["\sim"{description}, from=1-1, to=2-2]
    \arrow["\sim"{description}, from=1-3, to=2-2]
  \end{cd}
  So this tells us that a morphism $\pi_1(X,x) \to G^L$
  can be thought of as a $G^L$ bundle on $X$ equipped with a connection.
  Using the de Rham space, 
  this is precisely a $G^L$ bundle over $X_\DR$.
  which brings me to the second comment.

  \item $BG^L$ here is the \emph{classifying stack of $G^L$ bundles}.
  \footnote{
    If anyone asks : 
    give intuition of (fpqc) stack quotient.
  }
  It does what you expect it to do : 
  a map $S \to BG^L$ from a scheme $S$ is precisely a diagram
  of the form 
  \begin{cd}
    P & {\PT} \\
    {S}
    \arrow["{\text{$G^L$ bundle}}"', from=1-1, to=2-1]
    \arrow["{\text{$G^L$ equiv}}", from=1-1, to=1-2]
  \end{cd}
  Unfortunately, 
  we're short on time for a good motivation for stacks.
  One thing we need to know is :
  there is a point $\pi : \PT \to BG^L$ corresponding to
  the trivial $G^L$ bundle on $\PT$ and pullback of quasi-coherent sheaves
  along $\pi^*$ induces 
  \[
    \pi^* : \QCOH BG^L \map{\sim}{} \REP_k G^L
  \]

  \item (Technical remark. Only mention if anyone asks.)
  The $\HOM$ here is internal hom in $\PSTK_k$,
  the $\infty$-category of prestacks over $k$.
  This is the correct setting which reconciles the two
  conflicting goals : taking quotients and wanting descent.
  
\end{enumerate}

\end{document}