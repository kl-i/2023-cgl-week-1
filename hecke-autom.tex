\documentclass[./main.tex]{subfiles}
\begin{document}

To give a statement of the ``Galois to automorphic'' direction of 
categorical geometric Langlands,
I need to describe the two Hecke actions on $D\MOD(\BUN_G)$.
We begin with the automorphic side by
giving the geometric analogue of the local spherical Hecke algebras
from arithmetic Langlands.
Let $x \in X(k)$.

\begin{dfn}

%   The \emph{local Hecke correspondence at $x$} is defined by
% \begin{cd}
%   {\BUN_G D_x} & {\HECKE^{\LOC}_x} \\
% 	{\BUN_G D_x^\circ} & {\BUN_G D_x}
% 	\arrow[from=1-1, to=2-1]
% 	\arrow[from=2-2, to=2-1]
% 	\arrow[from=1-2, to=1-1]
% 	\arrow[from=1-2, to=2-2]
% 	\arrow["\lrcorner"{anchor=center, pos=0.125, rotate=-90}, draw=none, from=1-2, to=2-1]
% \end{cd}
The \emph{spherical Hecke category at $x$} 
is defined as \[
  \SPH_{G , x} := D\MOD( G(\cO^\wedge_x) \backslash G(K^\wedge_x) / G(\cO^\wedge_x) )
\]
where $ G(\cO^\wedge_x), G(K^\wedge_x)$
are the \emph{arc and loop group at $x$}.
\end{dfn}
The arc and loop groups are respectively schemes and ind-affine schemes
over $k$.
\cite[Proposition 1.3.2]{Zhu-16}
Their $k$-valued points recover the
honest-to-god groups $G(\cO^\wedge_x), G(K^\wedge_x)$.
The space underlying $\SPH_{G,x}$ is important so we'll give it a name.

\begin{dfn}
  We will call the (fpqc) double quotient 
\[
  \HECKE_x^\LOC := G(\cO^\wedge_x) \backslash G(K^\wedge_x) / G(\cO^\wedge_x)
\]
the \emph{local Hecke correspondence at $x$}.
The multiplication on the loop group induces a diagram
involving $\HECKE_x^\LOC$
\begin{cd}
  {G(\cO^\wedge_x) \backslash G(K^\wedge_x) / G(\cO^\wedge_x)} & {G(\cO^\wedge_x) \backslash G(K^\wedge_x) \times^{G(\cO^\wedge_x)} G(K^\wedge_x) / G(\cO^\wedge_x)} & {G(\cO^\wedge_x) \backslash G(K^\wedge_x) / G(\cO^\wedge_x)} \\
	& {G(\cO^\wedge_x) \backslash G(K^\wedge_x) / G(\cO^\wedge_x)}
	\arrow["{p_0}"', from=1-2, to=1-1]
	\arrow["{p_1}"', from=1-2, to=2-2]
	\arrow["m", from=1-2, to=1-3]
\end{cd}
% Using the same ``transition map'' argument we did for 
% vector bundles on $\bP^1$, 
% one can show that $\HECKE^\LOC_x$
% fits in a cartesian square
% \begin{cd}
%   {\BUN_G D_x} & {\HECKE^\LOC_x} \\
%   {\BUN_G D^\circ_x} & {\BUN_G D_x}
%   \arrow[from=1-1, to=2-1]
%   \arrow[from=2-2, to=2-1]
%   \arrow[from=1-2, to=1-1]
%   \arrow[from=1-2, to=2-2]
%   \arrow["\lrcorner"{anchor=center, pos=0.125, rotate=-90}, draw=none, from=1-2, to=2-1]
% \end{cd}
This allows one to define a monoidal product $\_ * \_$ 
on $\SPH_{G,x}$ via convolution.
\end{dfn}
% One can give a double coset description of $\HECKE_x^\LOC$
% by doing similar argument as what we did with vector bundles on $\bP^1$ :
% All $G$-bundles on the disk $D_x$ are trivial.
% So one obtain 
% \begin{prop}

%   \[
%     \HECKE_x^\LOC \simeq 
%     G(\cO^\wedge_x) \backslash G(K^\wedge_x) / G(\cO^\wedge_x)
%   \]
%   where $ G(\cO^\wedge_x), G(K^\wedge_x)$
%   are the \emph{arc and loop group at $x$},
%   suitably defined as presheaves over $k$.
% \end{prop}
% Pulling the cartesian square defining $\HECKE_x^\LOC$ along
% $\BUN_G X\setminus\set{x}$ gives :
% \begin{dfn}
  
%   The \emph{Hecke correspondence at $x$} is defined by :
%   \begin{cd}
%     {\BUN_G D_x} & {\HECKE^\LOC_x} \\
%     {\BUN_G D_x^\circ} & {\BUN_G D_x} & {\BUN_G X} & {\HECKE_x} \\
%     && {\BUN_G X\setminus\set{x}} & {\BUN_G X}
%     \arrow[from=1-1, to=2-1]
%     \arrow[from=2-2, to=2-1]
%     \arrow[from=1-2, to=1-1]
%     \arrow[from=1-2, to=2-2]
%     \arrow["\lrcorner"{anchor=center, pos=0.125, rotate=-90}, draw=none, from=1-2, to=2-1]
%     \arrow[from=3-3, to=2-1]
%     \arrow[from=2-3, to=1-1]
%     \arrow[from=3-4, to=2-2]
%     \arrow[from=3-4, to=3-3]
%     \arrow[from=2-3, to=3-3]
%     \arrow[from=2-4, to=2-3, "{p_0}"]
%     \arrow[from=2-4, to=3-4, "{p_1}"]
%     \arrow[from=2-4, to=1-2, "{p}"']
%     \arrow["\lrcorner"{anchor=center, pos=0.125, rotate=-90}, draw=none, from=2-4, to=3-3]
%   \end{cd}
% \end{dfn}
The monoidal structure on $\SPH_{G , x}$ is a categorification
of the algebra structure on the local Hecke algebras from arithmetic Langlands.
In arithmetic Langlands, 
the local Hecke algebras act on the space of (unramified) cusp forms
by Hecke operators.
In analogy to this, 
we now define an action of $\SPH_{G , x}$ on $D\MOD(\BUN_G)$.
\[
  H_x : \SPH_{G , x} \times D\MOD(\BUN_G) \to D\MOD(\BUN_G)
\]
But to do this, we need to know how to
make $D$-modules on $\BUN_G$ using a point $x \in X(k)$.
We have the following 
\begin{dfn}[Full level structure at $x$]
  
  Define the \emph{stack of $G$ bundles on $X$ equipped with
  full level structure at $x$} by the cartesian diagram 
  \begin{cd}
    {\BUN_G X} & {\BUN_{G,x}^\LVL} \\
    {\BUN_G D_x} & {\SPEC k}
    \arrow[from=1-1, to=2-1]
    \arrow["{\text{triv}}", from=2-2, to=2-1]
    \arrow[from=1-2, to=1-1]
    \arrow[from=1-2, to=2-2]
    \arrow["\lrcorner"{anchor=center, pos=0.125, rotate=-90}, draw=none, from=1-2, to=2-1]
  \end{cd}
\end{dfn}
There is a map 
\[
  \BUN_{G,x}^\LVL \to \BUN_G
\]
obtained by forgeting the full level structure at $x$.
This morphism is equivariant w.r.t. the action of
$G(\cO^\wedge_x)$ on $\BUN_{G,x}^\LVL$
and in fact is a $G(\cO^\wedge_x)$ bundle.
We obtain \[
  \BUN_G \simeq G(\cO^\wedge_x) \backslash \BUN_{G,x}^\LVL
\]
Thus, $D$-modules on $\BUN_G$ can be obtained from
arc-group-equivariant $D$-modules on $\BUN_{G,x}^\LVL$.
The action of $\SPH_{G,x}$ will also be defined by convolution.
With that idea,
we see that we need to extend the action of the arc group
to an action of the loop group.

\begin{dfn}[Regluing]
  There is an action of $G(K^\wedge_x)$ on $\BUN_G^x$ 
  \[
    \REGLUE : \BUN_{G,x}^\LVL \times G(K^\wedge_x)  \to \BUN_{G,x}^\LVL
  \]
  given intuitively by the following : 
  For $g \in G(K^\wedge_x)$, $P$ a $G$ bundle on $X$ and
  $s : \res{\TRIV}{D_x} \to \res{P}{D_x}$,
  make the following descent data : 
  \begin{cd}
    {\res{\TRIV}{D_x}} & {\res{\TRIV}{D_x^\circ}} & {\res{P}{D_x^\circ}} & {\res{P}{X\setminus\set{x}}}
    \arrow[from=1-2, to=1-1]
    \arrow[from=1-3, to=1-4]
    \arrow["{s g}", from=1-2, to=1-3]
  \end{cd}
  This then $\REGLUE(g , P , s)$ is defined as
  the $G$ bundle on $X$ obtained by gluing the above descent data.
  \footnote{
    If anyone asks : 
    Technically speaking,
    we must be able to do this gluing in families.
    Although fpqc descent is not enough to glue over
    non locally finite type spaces,
    the local finite type condition of $X$ ensures
    that we only need to know how to glue in finite type situations.
    This is then covered by fpqc descent.
    Alternatively, if one doesn't care about being minimal,
    one can use the Beauville--Laszlo theorem.
  }
\end{dfn}
We can now define the local Hecke action by convolution.
(Maybe skip the diagram.)

\begin{dfn}[Local Hecke action]
  
  Consider the diagram 
  \begin{cd}
    {G(K^\wedge_x)} & {G(K^\wedge_x) \times \BUN_{G,x}^\LVL} & {\BUN_{G,x}^\LVL} \\
    & {\BUN_{G,x}^\LVL}
    \arrow["{p_0}"', from=1-2, to=1-1]
    \arrow["{p_1}"', from=1-2, to=2-2]
    \arrow["{\REGLUE}", from=1-2, to=1-3]
  \end{cd}
  We can take a lot of (fpqc) quotients by the arc group and get 
  \begin{cd}
    {G(\cO^\wedge_x) \backslash G(K^\wedge_x) / G(\cO^\wedge_x)} 
    & {G(\cO^\wedge_x) \backslash G(K^\wedge_x) \times^{G(\cO^\wedge_x)} \BUN_{G,x}^\LVL} 
    & {G(\cO^\wedge_x) \backslash \BUN_{G,x}^\LVL} & {\BUN_G} \\
    & {G(\cO^\wedge_x) \backslash \BUN_{G,x}^\LVL} \\
    & {\BUN_G}
    \arrow["{p_0}"', from=1-2, to=1-1]
    \arrow["{p_1}"', from=1-2, to=2-2]
    \arrow["{\REGLUE}", from=1-2, to=1-3]
    \arrow["\simeq"{marking}, draw=none, from=2-2, to=3-2]
    \arrow["\simeq"{marking}, draw=none, from=1-3, to=1-4]
  \end{cd}
  The \emph{local Hecke action} is defined as\footnote{
    if anyone asks :
    give nice interpretation of middle term as
    Hecke correspondence at $x$.
  }
  \begin{align*}
    \SPH_{G,x} \times D\MOD(\BUN_G) \to D\MOD(\BUN_G) \\
    \cS , \cF \mapsto H_x(\cS , \cF) := (\REGLUE)_*\brkt{p_0^! S \otimes p_1^! \cF}
  \end{align*}
\end{dfn}

In order to make the definition of Hecke eigensheaf,
we need to bundle of the local Hecke actions together
into a single action over $X$.
This is roughly done in the following steps : 
(Skip all diagram and say ``replace $x$ with $X$''. 
Except for final formula.)
\begin{enumerate}
  \item 
  The arc and loop groups relativises to group presheaves
  over $X$ 
  \begin{cd}
    {G(\cO^\wedge_X)} && {G(K^\wedge_X)} \\
    & X
    \arrow[from=1-1, to=2-2]
    \arrow[from=1-3, to=2-2]
    \arrow[from=1-1, to=1-3]
  \end{cd}
  The local Hecke correspondences also relativises
  \[
    \HECKE^\LOC_X := 
    G(\cO^\wedge_X) \backslash G(K^\wedge_X) / G(\cO^\wedge_X) \to X
  \]

  \item The family $\BUN_{G,x}^\LVL \to \BUN_G$ relativises to
  \[
    \BUN_G^\LVL \to X \times \BUN_G
  \]
  and the action so does the action of the loop group \[
    G(K^\wedge_X) \times_X \BUN_G^\LVL \to \BUN_G^\LVL
  \]
  \item The convolution diagram relativises : 
  \begin{cd}
    {G(\cO^\wedge_X) \backslash G(K^\wedge_X) / G(\cO^\wedge_X)} 
    & {G(\cO^\wedge_X) \backslash G(K^\wedge_X) \times^{G(\cO^\wedge_X)} \BUN_{G,X}^\LVL} 
    & {G(\cO^\wedge_X) \backslash \BUN_{G,X}^\LVL} & {X \times \BUN_G} \\
    & {G(\cO^\wedge_X) \backslash \BUN_{G,X}^\LVL} \\
    & {\BUN_G}
    \arrow["{p_0}", from=1-2, to=1-1]
    \arrow["{p_1}"', from=1-2, to=2-2]
    \arrow["{\REGLUE}"', from=1-2, to=1-3]
    \arrow[from=2-2, to=3-2]
    \arrow["\simeq"{marking}, draw=none, from=1-3, to=1-4]
  \end{cd}
  This defines the \emph{global Hecke action} : 
  \footnote{
    If anyone asks : 
    give geometric interpretation of middle term as
    Hecke correspondence.
  }
  \begin{align*}
    \SPH_{G,X} \times D\MOD(\BUN_G) \to D\MOD(X \times \BUN_G) \\
    \cS , \cF \mapsto H(\cS , \cF) := (p_1)_*\brkt{p^! S \otimes p_0^! \cF}
  \end{align*}
\end{enumerate}


\end{document}