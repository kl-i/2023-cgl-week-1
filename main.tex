\documentclass{article}
\input{preamble.tex}

\addbibresource{mybib.bib}

\begin{document}

\title{The ``Galois to automorphic'' direction of 
  categorical geometric Langlands
}

\author{Ken Lee}
\date{Spring 2023}
\maketitle

\subfile{abstract.tex}

\tableofcontents

% Goals :
% \begin{enumerate}
%   \item one hour talk (probably max 10 pages)
%   \item the statement of (a version of)
%   the two side of categorical geometric Langlands
%   \item ``Hecke paradigm'' : BB localisation as a way of
%   giving D-modules on $\BUN_G$ from representations of
%   affine Kac--Moody Lie algebra at critical level.
%   \item Advice from Ruben: 
%   nothing beats doing maths by yourself, by hand.
%   As soon as you try to do something with details,
%   people tend to fall asleep.
%   I have experienced this.
%   E.g. Kahzdan's talk.
% \end{enumerate}

% \section{Geometric? Categorical? (5min)}

In a nutshell :
\begin{itemize}
  \item \emph{geometric} means instead of a number field $F$
  we have the field of rational functions $K := K(X)$ of some
  smooth, projective curve over some base field $k$,
  which we will assume to be $\bC$ in this talk.
  \item \emph{categorical} means that the focus is not on
  cusp forms but instead their categorification as sheaves on $\BUN_G$.
\end{itemize}

% Here's the setup :
% \begin{itemize}
%   \item We work over a fixed base field $k$.
%   ``Schemes'' will refer to schemes over $k$.
%   \item Fix $G$ a reductive group over $k$ and $G^L$ denote 
%   its Langlands dual.
%   \item We fix $X$ a smooth projective curve over a field $k$.
%   \item $K := K(X)$ refers to the field of rational functions of $X$.
%   \item There is an equivalence between the following three :
%   \begin{enumerate}
%     \item closed points $x \in X$
%     \item valuation rings inside $K$.
%     They turn out to be all discrete due to smoothness of $X$
%     \item places of $X$.
%     They are all non-archimedean and discrete.
%   \end{enumerate}
%   For closed points $x \in X$, 
%   \begin{itemize}
%     \item $v_x$ denotes to place of $K$ associated to $x$.
%     For $f \in K$, $v_x(f)$ is the order of $f$ at $x$.
%     \item $\cO_x \subs K$ denotes the discrete valuation ring consisting of
%     $f$ with no pole at $x$.
%     \item $\mathfrak{m}_x \subs \cO_x$ is the maximal ideal of
%     $f$ vanishing at $x$.
%     \item $\cO^\wedge_x := \LIM_{n \in \bN} \cO_x / \mathfrak{m}_x^{n+1}$
%     denotes the completion of $\cO_x$ w.r.t. $\mathfrak{m}_x$.
%     We call $D_x := \SPEC \cO^\wedge_x$ the disk around $x$.
%     \item $D^\wedge_x := \COLIM_{n \in \bN} \SPEC \cO_x / \mathfrak{m}_x^{n+1}$ 
%     denotes the formal disk around $x$.
%     We have $\cO_x^\wedge \simeq \HOM(D^\wedge_x , \bA^1)$.
%     \item $K^\wedge_x := \cO^\wedge_x \otimes_{\cO_x} K$ is 
%     the local field at $x$.
%     It is equivalently the completion of $K$ w.r.t. $v_x$.
%     We call $D_x^\circ := \SPEC K^\wedge_x = D_x \setminus \set{x} $ the
%     punctured disk around $x$.
%   \end{itemize}
% \end{itemize}

% \section{Some foundational constructions (Ideal 15m. Actual 37m)}
% \subfile{setup.tex}

\section{Galois side}
% \begin{itemize}
%   \item Question : why $\mathrm{LocSys}_{G^\vee}$?
%   Answer :
%   $\pi_1(\eta , \bar{\eta}) := \GAL(\bar{K} / K) \to \GL_n$ replaced with
%   $\pi_1(X , \bar{\eta}) \to \GL_n$ replaced with
%   $\Pi_1 X \to \GL_n$ replaced with
%   vector bundles on $X$ equipped with flat connection.
% \end{itemize}
\subfile{galois.tex}

\section{Automorphic side}

% \begin{itemize}
%   \item Question : why $\BUN_G$? 
%   Answer : geometric intepretation of adeles.
%   \item Question : why $D$-modules on $\BUN_G$? 
%   Answer : $\ell$-adic sheaves categorify functions.
%   $D$-modules equivalent to $\ell$-adic sheaves via Riemann--Hilbert.
% \end{itemize}

\subfile{autom.tex}

\section{Hecke action from automorphic side}

\subfile{hecke-autom.tex}

\section{Hecke action from Galois side and Hecke eigensheaves}

% \begin{itemize}
%   \item Question : how to make sense of sheaves on $\BUN_G$ having eigenvalues
%   in a $G^\vee$ local system?
%   Answer : make analogy with modules over Hecke algebra.
%   $D\MOD(\BUN_G)$ is module over $\REP G^\vee$.
%   STILL DONT KNOW WITH CONFIDENCE
%   (See \cite[Section 5.2]{BD}?)
%   \item Question : how does $\REP G^\vee$ act on $D\MOD(\BUN_G)$?
%   Answer : geometric Satake and integral transform across Hecke stack
%   \footnote{Reference for twisted geometric Satake?}
%   \footnote{Mention aesthetics point : geometric Satake gives a
%   classification-free construction of $G^\vee$.}
%   \item Question : so what is a Hecke eigensheaf?
%   Answer : write in.
%   \item Question : what is the statement of categorical geometric Langlands?
%   Answer : I know it's 
%   \[
%     \QCOH(\mathrm{LocSys}_{G^\vee}) \simeq D\MOD(\BUN_G)
%   \]
%   but idk where the Hecke eigensheaves ``condition'' went.
% \end{itemize}

\subfile{eigensheaf.tex}

% \section{How to make Hecke Eigensheaves? (10min)}

% Main question : given a $G^\vee$ local system $E$ on $X$,
% how to make a Hecke eigensheaf $\cF$ on $X$ with eigenvalue $E$?

% \begin{itemize}
%   \item table of Hecke paradigm / BB localisation comparing
%   classical case and infinite dim case
%   \item Localisation functor 
%   \[ 
%     (\mathfrak{g}\otimes K^\wedge_x , G(\cO^\wedge_x))\MOD \to
%     D\MOD(\BUN_G)
%   \]
% \end{itemize}
% \begin{itemize}
%   \item Question : what is the deal with the quantized Hitchin system?
%   Answer : NOT CLEAR. I know $\HITCH$ is commutative subalgebra of
%   $\Ga(\BUN_G , D^\CRIT)$ and that $\SPEC \HITCH$ closed embeds into 
%   $\LOCSYS_{G^\vee}$ but how does this line up with localisation?

%   TODO : reread 2nd overview by Gaitsgory and intro in \cite{FG-06}.
%   \item Question : how does $\SPEC \HITCH$ relate to local system?
%   Answer : Feigin--Frenkel isomorphism.
%   Analogous to Harish-Chandra isomorphism
% \end{itemize}

\printbibliography

\end{document}