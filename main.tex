\documentclass{article}
\usepackage[left=1in,right=1in]{geometry}
\usepackage{subfiles}
\usepackage{amsmath, amssymb, stmaryrd, verbatim} % math symbols
\usepackage{amsthm} % thm environment
\usepackage{mdframed} % Customizable Boxes
\usepackage{hyperref,nameref,cleveref,enumitem} % for references, hyperlinks
\usepackage[dvipsnames]{xcolor} % Fancy Colours
\usepackage{mathrsfs} % Fancy font
\usepackage{bbm} % mathbb numerals
\usepackage{tikz, tikz-cd, float} % Commutative Diagrams
\usetikzlibrary{decorations.pathmorphing} % for squiggly arrows in tikzcd
\usepackage{perpage}
\usepackage{parskip} % So that paragraphs look nice
\usepackage{ifthen,xargs} % For defining better commands
\usepackage[T1]{fontenc}
\usepackage[utf8]{inputenc}
\usepackage{tgpagella}
\usepackage{cancel}

% Bibliography
% \usepackage{url}
\usepackage[backend=biber,
            isbn=false,
            doi=false,
            giveninits=true,
            style=alphabetic,
            useprefix=true,
            maxcitenames=4,
            maxbibnames=4,
            sorting=nyt,
            citestyle=alphabetic]{biblatex}

% Shortcuts

% % Local to this project
\renewcommand\labelitemi{--} % Makes itemize use dashes instead of bullets
\newcommand{\bbrkt}[1]{\llbracket #1 \rrbracket}
\newcommand{\IND}{\mathrm{Ind}}
\newcommand{\AFF}{\mathrm{Aff}}
\newcommand{\FSCH}{\mathrm{fSch}}
\newcommand{\SCH}{\mathrm{Sch}}
\DeclareMathOperator{\SPEC}{Spec}
\DeclareMathOperator{\QCOH}{QCoh}
\DeclareMathOperator{\INDCOH}{IndCoh}
\newcommand{\DR}{\mathrm{dR}}
\newcommand{\SYM}{\mathrm{Sym}}
\newcommand{\HOM}{\underline{\mathrm{Hom}}}
\newcommand{\CALG}{\mathrm{cAlg}}
\DeclareMathOperator{\GL}{GL}
\DeclareMathOperator{\LIE}{Lie}
\newcommand{\AD}{\mathrm{Ad}}
\DeclareMathOperator{\MAX}{Max}
\DeclareMathOperator{\FROB}{Frob}
\newcommand{\SEP}{\mathrm{sep}}
\newcommand{\LOC}{\mathrm{Loc}}
\DeclareMathOperator{\PERV}{Perv}
\newcommand{\RED}{\mathrm{red}}
\newcommand{\ZAR}{\mathrm{Zar}}
\newcommand{\FPPF}{\mathrm{fppf}}
\newcommand{\ET}{\text{ét}}
\DeclareMathOperator{\SPF}{Spf}
\newcommand{\CTS}{\mathrm{cts}}
\newcommand{\INF}{\mathrm{Inf}}
\DeclareMathOperator{\STK}{Stk}
\DeclareMathOperator{\PSTK}{PStk}
\DeclareMathOperator{\SSET}{sSet}
\DeclareMathOperator{\BUN}{Bun}
\DeclareMathOperator{\PSH}{PSh}
\DeclareMathOperator{\SH}{Sh}
\newcommand{\FPQC}{\mathrm{fpqc}}
\DeclareMathOperator{\SPH}{Sph}
\newcommand{\FT}{\mathrm{ft}}
\DeclareMathOperator{\CRYS}{Crys}
\DeclareMathOperator{\GR}{Gr}
\DeclareMathOperator{\REP}{Rep}
\newcommand{\HITCH}{\mathrm{Hitch}}
\newcommand{\CRIT}{\mathrm{crit}}
\DeclareMathOperator{\LOCSYS}{LocSys}
\newcommand{\GRPD}{\mathrm{Grpd}}
\DeclareMathOperator{\MAP}{Map}
\DeclareMathOperator{\FUN}{Fun}
\newcommand{\HECKE}{\mathrm{Hecke}}
\newcommand{\TRIV}{\mathrm{Triv}}
\newcommand{\REGLUE}{\mathrm{ReGlue}}
\newcommand{\LVL}{\mathrm{lvl}}
\newcommand{\PT}{\mathrm{pt}}
\DeclareMathOperator{\COH}{\COH}


% % Misc
\newcommand{\brkt}[1]{\left(#1\right)}
\newcommand{\sqbrkt}[1]{\left[#1\right]}
\newcommand{\dash}{\text{-}}

% % Logic
\renewcommand{\implies}{\Rightarrow}
\renewcommand{\iff}{\Leftrightarrow}
\newcommand{\limplies}{\Leftarrow}
\newcommand{\NOT}{\neg\,}
\newcommand{\AND}{\, \land \,}
\newcommand{\OR}{\, \lor \,}
\newenvironment{forward}{($\implies$)}{}
\newenvironment{backward}{($\limplies$)}{}

% % Sets
\DeclareMathOperator{\supp}{supp}
\newcommand{\set}[1]{\left\{#1\right\}}
\newcommand{\st}{\,|\,}
\newcommand{\minus}{\setminus}
\newcommand{\subs}{\subseteq}
\newcommand{\ssubs}{\subsetneq}
\newcommand{\sups}{\supseteq}
\newcommand{\ssups}{\supset}
\DeclareMathOperator{\im}{Im}
\newcommand{\nothing}{\varnothing}
\DeclareMathOperator{\join}{\sqcup}
\DeclareMathOperator{\meet}{\sqcap}

% % Greek 
\newcommand{\al}{\alpha}
\newcommand{\be}{\beta}
\newcommand{\ga}{\gamma}
\newcommand{\de}{\delta}
\newcommand{\ep}{\varepsilon}
\newcommand{\ph}{\varphi}
\newcommand{\io}{\iota}
\newcommand{\ka}{\kappa}
\newcommand{\la}{\lambda}
\newcommand{\om}{\omega}
\newcommand{\si}{\sigma}

\newcommand{\Ga}{\Gamma}
\newcommand{\De}{\Delta}
\newcommand{\Th}{\Theta}
\newcommand{\La}{\Lambda}
\newcommand{\Si}{\Sigma}
\newcommand{\Om}{\Omega}

% % Mathbb
\newcommand{\bA}{\mathbb{A}}
\newcommand{\bB}{\mathbb{B}}
\newcommand{\bC}{\mathbb{C}}
\newcommand{\bD}{\mathbb{D}}
\newcommand{\bE}{\mathbb{E}}
\newcommand{\bF}{\mathbb{F}}
\newcommand{\bG}{\mathbb{G}}
\newcommand{\bH}{\mathbb{H}}
\newcommand{\bI}{\mathbb{I}}
\newcommand{\bJ}{\mathbb{J}}
\newcommand{\bK}{\mathbb{K}}
\newcommand{\bL}{\mathbb{L}}
\newcommand{\bM}{\mathbb{M}}
\newcommand{\bN}{\mathbb{N}}
\newcommand{\bO}{\mathbb{O}}
\newcommand{\bP}{\mathbb{P}}
\newcommand{\bQ}{\mathbb{Q}}
\newcommand{\bR}{\mathbb{R}}
\newcommand{\bS}{\mathbb{S}}
\newcommand{\bT}{\mathbb{T}}
\newcommand{\bU}{\mathbb{U}}
\newcommand{\bV}{\mathbb{V}}
\newcommand{\bW}{\mathbb{W}}
\newcommand{\bX}{\mathbb{X}}
\newcommand{\bY}{\mathbb{Y}}
\newcommand{\bZ}{\mathbb{Z}}

% % Mathcal
\newcommand{\cA}{\mathcal{A}}
\newcommand{\cB}{\mathcal{B}}
\newcommand{\cC}{\mathcal{C}}
\newcommand{\cD}{\mathcal{D}}
\newcommand{\cE}{\mathcal{E}}
\newcommand{\cF}{\mathcal{F}}
\newcommand{\cG}{\mathcal{G}}
\newcommand{\cH}{\mathcal{H}}
\newcommand{\cI}{\mathcal{I}}
\newcommand{\cJ}{\mathcal{J}}
\newcommand{\cK}{\mathcal{K}}
\newcommand{\cL}{\mathcal{L}}
\newcommand{\cM}{\mathcal{M}}
\newcommand{\cN}{\mathcal{N}}
\newcommand{\cO}{\mathcal{O}}
\newcommand{\cP}{\mathcal{P}}
\newcommand{\cQ}{\mathcal{Q}}
\newcommand{\cR}{\mathcal{R}}
\newcommand{\cS}{\mathcal{S}}
\newcommand{\cT}{\mathcal{T}}
\newcommand{\cU}{\mathcal{U}}
\newcommand{\cV}{\mathcal{V}}
\newcommand{\cW}{\mathcal{W}}
\newcommand{\cX}{\mathcal{X}}
\newcommand{\cY}{\mathcal{Y}}
\newcommand{\cZ}{\mathcal{Z}}

% % Mathfrak
\newcommand{\f}[1]{\mathfrak{#1}}

% % Mathrsfs
\newcommand{\s}[1]{\mathscr{#1}}

% % Category Theory
\DeclareMathOperator{\obj}{Obj}
\DeclareMathOperator{\END}{End}
\DeclareMathOperator{\AUT}{Aut}
\newcommand{\CAT}{\mathbf{Cat}}
\newcommand{\SET}{\mathbf{Set}}
\newcommand{\TOP}{\mathbf{Top}}
\newcommand{\MON}{\mathbf{Mon}}
\newcommand{\GRP}{\mathbf{Grp}}
\newcommand{\AB}{\mathbf{Ab}}
\newcommand{\RING}{\mathbf{Ring}}
\newcommand{\CRING}{\mathbf{CRing}}
\newcommand{\MOD}{\mathbf{Mod}}
\newcommand{\VEC}{\mathbf{Vec}}
\newcommand{\ALG}{\mathbf{Alg}}
\newcommand{\ORD}{\mathbf{Ord}}
\newcommand{\POSET}{\mathbf{PoSet}}
\newcommand{\id}{\mathbbm{1}}
\newcommand{\map}[2]{\yrightarrow[#2][2.5pt]{#1}[-1pt]}
\newcommand{\iso}[1][]{\cong_{#1}}
\newcommand{\OP}{\mathrm{op}}
\newcommand{\darrow}{\downarrow}
\newcommand{\LIM}{\varprojlim}
\newcommand{\COLIM}{\varinjlim}
\DeclareMathOperator{\coker}{coker}
\newcommand{\fall}[2]{\downarrow_{#2}^{#1}}
\newcommand{\lift}[2]{\uparrow_{#1}^{#2}}

% % Algebra
\newcommand{\nsub}{\trianglelefteq}
\newcommand{\inv}{{-1}}
\newcommand{\dvd}{\,|\,}
\DeclareMathOperator{\ev}{ev}

% % Analysis
\newcommand{\abs}[1]{\left\vert #1 \right\vert}
\newcommand{\norm}[1]{\left\Vert #1 \right\Vert}
\renewcommand{\bar}[1]{\overline{#1}}
\newcommand{\<}{\langle}
\renewcommand{\>}{\rangle}
\renewcommand{\hat}[1]{\widehat{#1}}
\renewcommand{\check}[1]{\widecheck{#1}}
\newcommand{\dsum}[2]{\sum_{#1}^{#2}}
\newcommand{\dprod}[2]{\prod_{#1}^{#2}}
\newcommand{\del}[2]{\frac{\partial#1}{\partial#2}}
\newcommand{\res}[2]{{% we make the whole thing an ordinary symbol
  \left.\kern-\nulldelimiterspace % automatically resize the bar with \right
  #1 % the function
  %\vphantom{\big|} % pretend it's a little taller at normal size
  \right|_{#2} % this is the delimiter
  }}

% % Galois
\DeclareMathOperator{\GAL}{Gal}
\DeclareMathOperator{\ORB}{Orb}
\DeclareMathOperator{\STAB}{Stab}
\newcommand{\emb}[3]{\mathrm{Emb}_{#1}(#2, #3)}
\newcommand{\Char}[1]{\mathrm{Char}#1}

%% code from mathabx.sty and mathabx.dcl to get some symbols from mathabx
\DeclareFontFamily{U}{mathx}{\hyphenchar\font45}
\DeclareFontShape{U}{mathx}{m}{n}{
      <5> <6> <7> <8> <9> <10>
      <10.95> <12> <14.4> <17.28> <20.74> <24.88>
      mathx10
      }{}
\DeclareSymbolFont{mathx}{U}{mathx}{m}{n}
\DeclareFontSubstitution{U}{mathx}{m}{n}
\DeclareMathAccent{\widecheck}{0}{mathx}{"71}

% Arrows with text above and below with adjustable displacement
% (Stolen from Stackexchange)
\newcommandx{\yaHelper}[2][1=\empty]{
\ifthenelse{\equal{#1}{\empty}}
  % no offset
  { \ensuremath{ \scriptstyle{ #2 } } } 
  % with offset
  { \raisebox{ #1 }[0pt][0pt]{ \ensuremath{ \scriptstyle{ #2 } } } }  
}

\newcommandx{\yrightarrow}[4][1=\empty, 2=\empty, 4=\empty, usedefault=@]{
  \ifthenelse{\equal{#2}{\empty}}
  % there's no text below
  { \xrightarrow{ \protect{ \yaHelper[ #4 ]{ #3 } } } } 
  % there's text below
  {
    \xrightarrow[ \protect{ \yaHelper[ #2 ]{ #1 } } ]
    { \protect{ \yaHelper[ #4 ]{ #3 } } } 
  } 
}

% xcolor
\definecolor{darkgrey}{gray}{0.10}
\definecolor{lightgrey}{gray}{0.30}
\definecolor{slightgrey}{gray}{0.80}
\definecolor{softblue}{RGB}{30,100,200}

% hyperref
\hypersetup{
      colorlinks = true,
      linkcolor = {softblue},
      citecolor = {blue}
}

\newcommand{\link}[1]{\hypertarget{#1}{}}
\newcommand{\linkto}[2]{\hyperlink{#1}{#2}}

% Perpage
\MakePerPage{footnote}

% Theorems

% % custom theoremstyles
\newtheoremstyle{definitionstyle}
{5pt}% above thm
{0pt}% below thm
{}% body font
{}% space to indent
{\bf}% head font
{\vspace{1mm}}% punctuation between head and body
{\newline}% space after head
{\thmname{#1}\thmnote{\,\,--\,\,#3}}

\newtheoremstyle{exercisestyle}%
{5pt}% above thm
{0pt}% below thm
{\it}% body font
{}% space to indent
{\it}% head font
{.}% punctuation between head and body
{ }% space after head
{\thmname{#1}\thmnote{ (#3)}}

\newtheoremstyle{examplestyle}%
{5pt}% above thm
{0pt}% below thm
{\it}% body font
{}% space to indent
{\it}% head font
{.}% punctuation between head and body
{\newline}% space after head
{\thmname{#1}\thmnote{ (#3)}}

\newtheoremstyle{remarkstyle}%
{5pt}% above thm
{0pt}% below thm
{}% body font
{}% space to indent
{\it}% head font
{.}% punctuation between head and body
{ }% space after head
{\thmname{#1}\thmnote{\,\,--\,\,#3}}

\newtheoremstyle{questionstyle}%
{5pt}% above thm
{0pt}% below thm
{}% body font
{}% space to indent
{\it}% head font
{?}% punctuation between head and body
{ }% space after head
{\thmname{#1}\thmnote{\,\,--\,\,#3}}

% Custom Environments

% % Theorem environments

\theoremstyle{definitionstyle}
\newmdtheoremenv[
    linewidth = 2pt,
    leftmargin = 0pt,
    rightmargin = 0pt,
    linecolor = darkgrey,
    topline = false,
    bottomline = false,
    rightline = false,
    footnoteinside = true
]{dfn}{Definition}
\newmdtheoremenv[
    linewidth = 2 pt,
    leftmargin = 0pt,
    rightmargin = 0pt,
    linecolor = darkgrey,
    topline = false,
    bottomline = false,
    rightline = false,
    footnoteinside = true
]{prop}{Proposition}
\newmdtheoremenv[
    linewidth = 2 pt,
    leftmargin = 0pt,
    rightmargin = 0pt,
    linecolor = darkgrey,
    topline = false,
    bottomline = false,
    rightline = false,
    footnoteinside = true
]{cor}{Corollary}

\theoremstyle{exercisestyle}
\newmdtheoremenv[
    linewidth = 0.7 pt,
    leftmargin = 20pt,
    rightmargin = 0pt,
    linecolor = darkgrey,
    topline = false,
    bottomline = false,
    rightline = false,
    footnoteinside = true
]{ex}{Exercise}
\newmdtheoremenv[
    linewidth = 0.7 pt,
    leftmargin = 20pt,
    rightmargin = 0pt,
    linecolor = darkgrey,
    topline = false,
    bottomline = false,
    rightline = false,
    footnoteinside = true
]{lem}{Lemma}

\theoremstyle{examplestyle}
\newmdtheoremenv[
    linewidth = 0.7 pt,
    leftmargin = 0pt,
    rightmargin = 0pt,
    linecolor = darkgrey,
    topline = false,
    bottomline = false,
    rightline = false,
    footnoteinside = true
]{eg}{Example}
\newmdtheoremenv[
    linewidth = 0.7 pt,
    leftmargin = 0pt,
    rightmargin = 0pt,
    linecolor = darkgrey,
    topline = false,
    bottomline = false,
    rightline = false,
    footnoteinside = true
]{ceg}{Counter Example}

\theoremstyle{remarkstyle}
\newtheorem{rmk}{Remark}

\theoremstyle{questionstyle}
\newtheorem{question}{Question}

\newenvironment{proof1}{
  \begin{proof}\renewcommand\qedsymbol{$\blacksquare$}
}{
  \end{proof}
} % Proofs ending with black qedsymbol 

% % tikzcd diagram 
\newenvironment{cd}{
    \begin{figure}[H]
    \centering
    \begin{tikzcd}
}{
    \end{tikzcd}
    \end{figure}
}

% tikzcd
% % Substituting symbols for arrows in tikz comm-diagrams.
\tikzset{
  symbol/.style={
    draw=none,
    every to/.append style={
      edge node={node [sloped, allow upside down, auto=false]{$#1$}}}
  }
}

\addbibresource{mybib.bib}

\begin{document}

\title{The ``Galois to automorphic'' direction of 
  categorical geometric Langlands
}

\author{Ken Lee}
\date{Spring 2023}
\maketitle

\subfile{abstract.tex}

\tableofcontents

% Goals :
% \begin{enumerate}
%   \item one hour talk (probably max 10 pages)
%   \item the statement of (a version of)
%   the two side of categorical geometric Langlands
%   \item ``Hecke paradigm'' : BB localisation as a way of
%   giving D-modules on $\BUN_G$ from representations of
%   affine Kac--Moody Lie algebra at critical level.
%   \item Advice from Ruben: 
%   nothing beats doing maths by yourself, by hand.
%   As soon as you try to do something with details,
%   people tend to fall asleep.
%   I have experienced this.
%   E.g. Kahzdan's talk.
% \end{enumerate}

% \section{Geometric? Categorical? (5min)}

In a nutshell :
\begin{itemize}
  \item \emph{geometric} means instead of a number field $F$
  we have the field of rational functions $K := K(X)$ of some
  smooth, projective curve over some base field $k$,
  which we will assume to be $\bC$ in this talk.
  \item \emph{categorical} means that the focus is not on
  cusp forms but instead their categorification as sheaves on $\BUN_G$.
\end{itemize}

% Here's the setup :
% \begin{itemize}
%   \item We work over a fixed base field $k$.
%   ``Schemes'' will refer to schemes over $k$.
%   \item Fix $G$ a reductive group over $k$ and $G^L$ denote 
%   its Langlands dual.
%   \item We fix $X$ a smooth projective curve over a field $k$.
%   \item $K := K(X)$ refers to the field of rational functions of $X$.
%   \item There is an equivalence between the following three :
%   \begin{enumerate}
%     \item closed points $x \in X$
%     \item valuation rings inside $K$.
%     They turn out to be all discrete due to smoothness of $X$
%     \item places of $X$.
%     They are all non-archimedean and discrete.
%   \end{enumerate}
%   For closed points $x \in X$, 
%   \begin{itemize}
%     \item $v_x$ denotes to place of $K$ associated to $x$.
%     For $f \in K$, $v_x(f)$ is the order of $f$ at $x$.
%     \item $\cO_x \subs K$ denotes the discrete valuation ring consisting of
%     $f$ with no pole at $x$.
%     \item $\mathfrak{m}_x \subs \cO_x$ is the maximal ideal of
%     $f$ vanishing at $x$.
%     \item $\cO^\wedge_x := \LIM_{n \in \bN} \cO_x / \mathfrak{m}_x^{n+1}$
%     denotes the completion of $\cO_x$ w.r.t. $\mathfrak{m}_x$.
%     We call $D_x := \SPEC \cO^\wedge_x$ the disk around $x$.
%     \item $D^\wedge_x := \COLIM_{n \in \bN} \SPEC \cO_x / \mathfrak{m}_x^{n+1}$ 
%     denotes the formal disk around $x$.
%     We have $\cO_x^\wedge \simeq \HOM(D^\wedge_x , \bA^1)$.
%     \item $K^\wedge_x := \cO^\wedge_x \otimes_{\cO_x} K$ is 
%     the local field at $x$.
%     It is equivalently the completion of $K$ w.r.t. $v_x$.
%     We call $D_x^\circ := \SPEC K^\wedge_x = D_x \setminus \set{x} $ the
%     punctured disk around $x$.
%   \end{itemize}
% \end{itemize}

% \section{Some foundational constructions (Ideal 15m. Actual 37m)}
% \subfile{setup.tex}

\section{Galois side}
% \begin{itemize}
%   \item Question : why $\mathrm{LocSys}_{G^\vee}$?
%   Answer :
%   $\pi_1(\eta , \bar{\eta}) := \GAL(\bar{K} / K) \to \GL_n$ replaced with
%   $\pi_1(X , \bar{\eta}) \to \GL_n$ replaced with
%   $\Pi_1 X \to \GL_n$ replaced with
%   vector bundles on $X$ equipped with flat connection.
% \end{itemize}
\subfile{galois.tex}

\section{Automorphic side}

% \begin{itemize}
%   \item Question : why $\BUN_G$? 
%   Answer : geometric intepretation of adeles.
%   \item Question : why $D$-modules on $\BUN_G$? 
%   Answer : $\ell$-adic sheaves categorify functions.
%   $D$-modules equivalent to $\ell$-adic sheaves via Riemann--Hilbert.
% \end{itemize}

\subfile{autom.tex}

\section{Hecke action from automorphic side}

\subfile{hecke-autom.tex}

\section{Hecke action from Galois side and Hecke eigensheaves}

% \begin{itemize}
%   \item Question : how to make sense of sheaves on $\BUN_G$ having eigenvalues
%   in a $G^\vee$ local system?
%   Answer : make analogy with modules over Hecke algebra.
%   $D\MOD(\BUN_G)$ is module over $\REP G^\vee$.
%   STILL DONT KNOW WITH CONFIDENCE
%   (See \cite[Section 5.2]{BD}?)
%   \item Question : how does $\REP G^\vee$ act on $D\MOD(\BUN_G)$?
%   Answer : geometric Satake and integral transform across Hecke stack
%   \footnote{Reference for twisted geometric Satake?}
%   \footnote{Mention aesthetics point : geometric Satake gives a
%   classification-free construction of $G^\vee$.}
%   \item Question : so what is a Hecke eigensheaf?
%   Answer : write in.
%   \item Question : what is the statement of categorical geometric Langlands?
%   Answer : I know it's 
%   \[
%     \QCOH(\mathrm{LocSys}_{G^\vee}) \simeq D\MOD(\BUN_G)
%   \]
%   but idk where the Hecke eigensheaves ``condition'' went.
% \end{itemize}

\subfile{eigensheaf.tex}

% \section{How to make Hecke Eigensheaves? (10min)}

% Main question : given a $G^\vee$ local system $E$ on $X$,
% how to make a Hecke eigensheaf $\cF$ on $X$ with eigenvalue $E$?

% \begin{itemize}
%   \item table of Hecke paradigm / BB localisation comparing
%   classical case and infinite dim case
%   \item Localisation functor 
%   \[ 
%     (\mathfrak{g}\otimes K^\wedge_x , G(\cO^\wedge_x))\MOD \to
%     D\MOD(\BUN_G)
%   \]
% \end{itemize}
% \begin{itemize}
%   \item Question : what is the deal with the quantized Hitchin system?
%   Answer : NOT CLEAR. I know $\HITCH$ is commutative subalgebra of
%   $\Ga(\BUN_G , D^\CRIT)$ and that $\SPEC \HITCH$ closed embeds into 
%   $\LOCSYS_{G^\vee}$ but how does this line up with localisation?

%   TODO : reread 2nd overview by Gaitsgory and intro in \cite{FG-06}.
%   \item Question : how does $\SPEC \HITCH$ relate to local system?
%   Answer : Feigin--Frenkel isomorphism.
%   Analogous to Harish-Chandra isomorphism
% \end{itemize}

\printbibliography

\end{document}